\documentclass[11pt]{beamer}
\usetheme{Pittsburgh}
\usepackage[utf8]{inputenc}
\usepackage{amsmath}
\usepackage{amsfonts}
\usepackage{amssymb}
\usepackage{wrapfig}
\usepackage{array}
\usepackage{color}
\author{Gordon Lawrenz, Oleg Chernikov, Alina Shigabutdinova, Yevhen Kuznietsov}
\title[Exercise 1]{Real Time Search in Microblogs: Exercise 1}
%\setbeamercovered{transparent} 
%\setbeamertemplate{navigation symbols}{} 
%\logo{} 
\institute{RWTH Aachen} 
%\date{} 
\subject{Real Time Search in Microblogs} 

\begin{document}
	
	\begin{frame}
	\titlepage
	\end{frame}
	
	%---------------------------------------------------------------------------
	
	\begin{frame}
		\tableofcontents
	\end{frame}
	
	%---------------------------------------------------------------------------
	
	\begin{frame}
		\frametitle{Real-time search}
        Search - process of information retrieval according to the information need\\
Real-time search aims at:
		\begin{itemize}
			\item
				Providing users with the most recent content
            \item
            	Collecting content from specific sources (such as social media)
		\end{itemize}
	\end{frame}
	
    %---------------------------------------------------------------------------
	
	\begin{frame}
		\frametitle{???}
        Today the speed of information appearance in the internet is so big that it becomes 		impossible to get all the latest changes immediately
		\begin{itemize}
			\item
				Example1
            \item
            	Example2
		\end{itemize}
	\end{frame}
	
    %---------------------------------------------------------------------------
	
    \begin{frame}
		\frametitle{Real-time search engines}
		\begin{itemize}
			\item
				Do not crawl for information, instead use either own api or social media source api in order to get direct feed
            \item
            	Do not aggregate the data as traditional search engines do
            \item
            	Indexing structures are optimized for real-time ranking
		\end{itemize}
	\end{frame}
	
    %---------------------------------------------------------------------------
    
    \begin{frame}
		\frametitle{Real-time search engines}
		\begin{itemize}
			\item
				OneRiot (closed)
            \item
            	Collecta (paused)
            \item
            	Google's "latest" (still not real-time)
            \item
            	SocialMention (slow, but results seem to be relevant)
            \item
            	Topsy (fast, result seem to be relevant)
            \item
            	Social searcher (not slow, many irrelevant results)
		\end{itemize}
	\end{frame}
	
    %---------------------------------------------------------------------------
    
    \begin{frame}
		\frametitle{Real-time search challenges}
		\begin{itemize}
			\item
				Spam filtering
            \item
            	Relevance of results
            \item
            	Commercial attractiveness
		\end{itemize}
	\end{frame}
	
    %---------------------------------------------------------------------------
    
    \begin{frame}
		\frametitle{Commercial attractiveness of real-time search}
        For search engines the main part of income usually accounts for online advertising (Google – 89.5 percent in 2014)\\
Issues:
		\begin{itemize}
			\item
				Popularity of real-time search
            \item
            	Specificity of real-time search queries
            \item
            	Pollution
		\end{itemize}
	\end{frame}
	
    %---------------------------------------------------------------------------
    
    \begin{frame}
		\frametitle{Perspectives}
        
	\end{frame}
	
    %---------------------------------------------------------------------------
    
    \begin{frame}
		\frametitle{Something about user behaviour or other statistics}
    \end{frame}
	
    %---------------------------------------------------------------------------
    
	\begin{frame}
		\frametitle{Related Work}
		\begin{enumerate}
			\item
				C. Chen, F. Li, B. C. Ooi, and S. Wu, “TI: An efficient indexing mechanism for real-time search on tweets," 2011
			\item
				M. Busch, K. Gade, B. Larson, P. Lok, S. Luckenbill, and J. Lin, "Earlybird: Real-time search at twitter," 2012
			\item
				 J. Yao, B. Cui, Z. Xue, and Q. Liu, "Provenance-based indexing support in micro-blog platforms," 2012
			\item
				A. Magdy, M. F. Mokbel, S. Elnikety, S. Nath, and Y. He., "Mercury: A Memory-Constrained Spatio-temporal Real-time Search on Microblogs", 2014
			\item
				A. Magdy, A. M. Aly, M. F. Mokbel, S. Elnikety, Y. He, S. Nath, "Mars: Real-time Spatio-temporal Queries on Microblogs", 2014
		\end{enumerate}
	\end{frame}
	
    %---------------------------------------------------------------------------
    
    %---------------------------------------------------------------------------
	
	\begin{frame}
		\frametitle{Research directions}
		\begin{itemize}
			\item
				Ranking microblogs with respect to queries
            \item
            	Indexing microblogs for efficient search
            \item
            	Analyzing the characteristics of microblog data and
queries
		\end{itemize}
	\end{frame}
	
    %---------------------------------------------------------------------------
	
	\begin{frame}
		\frametitle{TI: An efficient indexing mechanism for real-time search on tweets}
        \includegraphics[]{Tweet_Table}
        \begin{itemize}
			\item
            	Indexing approach
            \item
            	Reduction of indexing costs without loss of quality
            \item
            	\textcolor{red}{But:} relevant responses to the queries are not always included
        \end{itemize}
	\end{frame}
	
	%---------------------------------------------------------------------------
    
    %---------------------------------------------------------------------------
	
	\begin{frame}
		\frametitle{Earlybird: Real-time search at twitter}
           \begin{table}[h!]
             \begin{center}
               \begin{tabular}{ p{5cm}  c }
                  \begin{minipage}{5cm}
                    \begin{itemize}
                      \item High speed indexing and query evaluation
                      \item Use of concurrency
                      \item \textcolor{red}{But:} Relevance of the tweet is judged only by its freshness
                    \end{itemize}
                  \end{minipage}
                  &
                  \begin{minipage}{.5\textwidth}
                    \includegraphics[width=\linewidth, height=60mm]{Earlybird}
                  \end{minipage}
                \end{tabular}
              \end{center}
            \end{table}
	\end{frame}
	
	%---------------------------------------------------------------------------
    
    %---------------------------------------------------------------------------
	
	\begin{frame}
		\frametitle{Provenance-based indexing support in micro-blog platforms}
        \begin{table}[h!]
             \begin{center}
               \begin{tabular}{ p{5cm}  c }
                  \begin{minipage}{5cm}
                    \begin{itemize}
                      \item Search of the origin of the micro-blog
                      \item Pursues different goal though
                    \end{itemize}
                  \end{minipage}
                  &
                  \begin{minipage}{.5\textwidth}
                    \includegraphics[width=\linewidth, height=60mm]{provenance}
                  \end{minipage}
                \end{tabular}
              \end{center}
            \end{table}
	\end{frame}
	
	%---------------------------------------------------------------------------
	
	\begin{frame}
		\frametitle{Mercury: A Memory-Constrained Spatio-temporal Real-time Search on Microblogs}
		Mercury is a system for real-time support of \textcolor{red}{top-k} \textcolor{blue}{spatio-temporal queries} on microblogs, where users are able to browse recent microblogs near their locations.
		\begin{center}
			\includegraphics[scale=0.8]{mercury_arch.jpg}
		\end{center}
	\end{frame}
	
	%---------------------------------------------------------------------------
	
	\begin{frame}
		\frametitle{Mars: Real-time Spatio-temporal Queries on Microblogs}
		Mars uses the microblogs location information to support a wide variety of important \textcolor{blue}{spatio-temporal queries} on microblogs, including \textcolor{blue}{range, nearest-neighbor, and aggregate queries}. 
		\begin{center}
			\includegraphics[scale=0.7]{mars_arch.jpg}		
		\end{center}
	\end{frame}

\end{document}
